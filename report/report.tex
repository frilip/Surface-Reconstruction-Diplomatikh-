\documentclass[11pt, letterpaper]{article}
\usepackage{geometry}
\geometry{
 a4paper,
 total={170mm,257mm},
 left=20mm,
 top=20mm,
 }
\usepackage{graphicx} % Required for inserting images
\usepackage[english,greek]{babel}
\usepackage{mathabx} % Load the mathabx package
\usepackage{subcaption}
\usepackage{dirtytalk}
\usepackage{hyperref}
\usepackage{comment}
\usepackage{caption}
\usepackage{float}
\usepackage{indentfirst}
\newcommand{\en}{\selectlanguage{english}}
\newcommand{\gr}{\selectlanguage{greek}}

\captionsetup[subfigure]{width=0.7\textwidth}


\graphicspath{{../figures/}} % specify the path to the images

\title{\en Surface reconstruction methods}
\author{Φίλιππος Ρωσσίδης}
\date{\today}


\begin{document}
\maketitle

\section{Προγράμματα}

Χρησιμοποιώ:

\begin{itemize}
    \item \en \textbf{MeshLab} \gr για εκτίμηση \en normals, \gr και \en screened Poisson surface reconstruction. \gr 
    \item \en \textbf{CGAL} \gr για \en Kinetic, Advancing front
    \item \en \textbf{Blender} \gr για δημιουργία και επεξεργασία δεδομένων.
\end{itemize}



\section{Θεωρητικά: καλή και κακή τοπολογία στη γραφική}

Μια σύντομη αναφορά στην καλή τοπολογία στο \en context \gr της γραφικής, διότι 
συζητήσαμε για αυτό. 

Ορολογία: \en
\begin{itemize}
    \item \textbf{Quads}: \gr τετράπλευρα \en
    \item \textbf{Ngons}: \gr πολύγωνα με (συνήθως) παραπάνω από 4 πλευρές
\end{itemize}
\gr


Θέλουμε τα \en meshes \gr να αποτελούνται από \en quads, \gr και όχι τρίγωνα, ούτε 
\en ngons. \gr 

\begin{figure}
    \begin{subfigure}{0.3\textwidth}
        \centering
        \includegraphics[width=\linewidth]{ball_good_topo.png}
        \caption{Ο καλός}
        \label{}
    \end{subfigure}
    \begin{subfigure}{0.3\textwidth}
        \centering
        \includegraphics[width=\linewidth]{ball_bad_topo.png}
        \caption{ο κακός}
        \label{}
    \end{subfigure}
    \begin{subfigure}{0.3\textwidth}
        \centering
        \includegraphics[width=\linewidth]{ball_ugly_topo.png}
        \caption{και ο άσχημος}
        \label{}
    \end{subfigure}
    \caption{Σύγκριση τοπολογίας}
    \label{fig:good_bad_ugly}
\end{figure}


Αυτό διότι στα \en quads \gr (αν τοποθετηθούν σωστά) μπορούμε να εντοπίσουμε 
\en edge flow, \gr όπως π.χ. είναι χρωματισμένο με πορτοκαλί για 3 περιπτώσεις στο 
σχήμα \ref{fig:good_bad_ugly}. Στο αριστερά μπορούμε να βρούμε συνεχόμενες 
ροές ακμών σε όλο το μήκος της σφαίρας, στο μεσαίο (παρότι αποτελείται απο \en
quads) \gr υπάρχει ένα \say{εμπόδιο}, στο αριστερό επειδή η επιφάνεια αποτελείται
από τρίγωνα, δεν μπορεί πουθενά να εντοπιστεί ροή.

Ο λόγος που θέλουμε να υπάρχει ροή στις ακμές είναι γιατί το \en mesh \gr 
θα χρησιμοποιηθεί στη συνέχεια σε διάφορα \en pipelines \gr που δεν λειτουργούν 
καλά χωρίς αυτήν. 

Παραδείγματος χάριν, αν θέλουμε να κάνουμε \en animate \gr το \en mesh \gr 
(δηλαδή να του δώσουμε κίνηση) σκεφτείτε π.χ. τα βλέφαρα ενός χαρακτήρα τα οποία 
θέλουμε να ανοιγοκλείνουν, θα πρέπει να υπάρχει ξεκάθαρη ροή των βλεφάρων 
γύρω από το μάτι, ώστε να μπορούν να κινηθούν χωρίς να παραμορφωθεί η γεωμετρία τους.

Ένας άλλος λόγος φαίνεται στα σχήματα \ref{fig:brick_1}, \ref{fig:brick_2}.
Εδώ χρησιμοποιείται η τεχνική \en subdivision surface, \gr όπου υποδιαιρείται 
η επιφάνεια σε μικρότερα πολύγωνα (γίνεται σε κάθε βήμα διπλασιασμός του πλήθους
τους και \en interpolation \gr ανά 3 αν δεν κάνω λάθος) για να γίνει πιο 
\en smooth \gr το σχήμα. 

Στην αριστερά τοπολογία \en(quads), \gr η μεθοδος λειτουργεί κανονικά, 
ενώ στην δεξιά \en(Ngons), \gr το \en interpolation \gr αδυνατεί να λειτουργήσει
αφού κάποιοι κόμβοι έχουν γείτονες μόνο προς μία κατεύθυνση.


\begin{figure}
    \begin{subfigure}{0.5\textwidth}
        \centering
        \includegraphics[width=\linewidth]{brick_good_topo.png}
        \caption{Αποτελείται από τετράπλευρα \en (quads)}
        \label{}
    \end{subfigure}
    \begin{subfigure}{0.5\textwidth}
        \centering
        \includegraphics[width=\linewidth]{brick_bad_topo.png}
        \caption{Αποτελείται από \en ngons}
        \label{}
    \end{subfigure}
    \caption{Σύγκριση τοπολογίας πριν το \en subdivision}
    \label{fig:brick_1}
\end{figure}



\begin{figure}
    \begin{subfigure}{0.5\textwidth}
        \centering
        \includegraphics[width=\linewidth]{subdiv_good_topo.png}
        \caption{Αποτελείται από τετράπλευρα \en (quads)}
        \label{}
    \end{subfigure}
    \begin{subfigure}{0.5\textwidth}
        \centering
        \includegraphics[width=\linewidth]{subdiv_bad_topo.png}
        \caption{Αποτελείται από \en ngons}
        \label{}
    \end{subfigure}
    \caption{Σύγκριση τοπολογίας μετά το \en subdivision} 
    \label{fig:brick_2}
\end{figure}




\section{Δημιουργία \en point cloud \gr δεδομένων }

Η βιβλιοθήκη \en CGAL \gr με την εγκατάστασή της παρέχει κάποια \en point clouds \gr 
τα οποία χρησιμοποιώ για \en tests \gr των μεθόδων (π.χ. σχήμα \ref{fig:point_clouds}).

Τα συμπεριλαμβάνω στο φάκελο \en CGAL-6.1 \gr

\begin{figure}[h!]
    \begin{subfigure}{0.5\textwidth}
        \centering
        \includegraphics[width=0.7\linewidth]{kitten_CGAL.png}
        \caption{Σημεία μοντέλου γάτας, χωρίς θόρυβο, με σταθερή πυκνότητα.}
        \label{fig:kitten_points}
    \end{subfigure}
    \begin{subfigure}{0.5\textwidth}
        \centering
        \includegraphics[width=0.7\linewidth]{building_large.png}
        \caption{Λεπτομερές κτήριο με θόρυβο και ελλειπή δεδομένα.}
        \label{}
    \end{subfigure}
    \caption{Παραδείγματα \en point data \gr της \en CGAL. }
    \label{fig:point_clouds}
\end{figure}


Επίσης δημιουργώ δικά μου δεδομένα με χρήση του \en Blender: \gr

Μοντελοποιώ ένα \en mesh, \gr και με τα \en geometry nodes \gr (μέθοδος \en procedural \gr παραγωγής
γεωμετριών) του σχήματος \ref{fig:geometry_nodes}, κάνω \en scatter \gr σημεία στην 
επιφάνεια του \en mesh \gr με τυχαίο τρόπο. Μπορώ να ελέγξω την πυκνότητα με 
την τιμή \en density. \gr 

Μπορώ να το επεξεργαστώ στην πορεία για να συμπεριλάβω θόρυβο κλπ.


\begin{figure}
    \centering
    \includegraphics[width=0.7\linewidth]{geometry_nodes_pointcloud_gen.png}
    \caption{\en Scatter points on surface}
    \label{fig:geometry_nodes}
\end{figure}

\begin{figure}
    \begin{subfigure}{0.3\textwidth}
        \centering
        \includegraphics[width=\linewidth]{blender_default_cube.png}
        \caption{\en mesh}
        \label{}
    \end{subfigure}
    \begin{subfigure}{0.3\textwidth}
        \centering
        \includegraphics[width=\linewidth]{cude_density_10.png}
        \caption{\en density = 10 }
        \label{}
    \end{subfigure}
    \begin{subfigure}{0.3\textwidth}
        \centering
        \includegraphics[width=\linewidth]{cube_density_200.png}
        \caption{\en density = 200}
        \label{}
    \end{subfigure}
    \caption{}
    \label{}
\end{figure}


\section{\en Delaunay-like methods}

Μέθοδοι που χρησιμοποιούν \en delaunay \gr σε κάποιο στάδιό τους.
Χρησιμοποιούν ακριβώς τα σημεία του \en point cloud \gr ως κόμβους του \en mesh. \gr


\subsection{\en Advancing front}

Προς το παρών, δεν ασχολούμαι με το πως λειτουργεί ο αλγόριθμος, όσο με τα αποτελέσματα
που παράγει.

Χρησιμοποιώ τη συνάρτηση που παρέχει η βιβλιοθήκη \en CGAL \gr (συμπεριλαμβάνω και
τα αρχεία κώδικα).
Εφαρμόζοντας στο \en point set \gr \ref{fig:kitten_points} παίρνουμε
ένα πολύ καλό αποτέλεσμα (σχήμα \ref{fig:kitten_adv_front}) επειδή τα δεδομένα 
είχαν σταθερή και επαρκή πυκνότητα (ούτε πολύ λεπτομερή, ούτε ελλειπή) και δεν 
είχαν θόρυβο.

\begin{figure}
    \begin{subfigure}{0.5\textwidth}
        \centering
        \includegraphics[width=0.5\linewidth]{kitten_Advancing_front.png}
        \caption{\en kitten, Advancing front}
        \label{fig:kitten_adv_front}
    \end{subfigure}
    \begin{subfigure}{0.5\textwidth}
        \centering
        \includegraphics[width=0.5\linewidth]{cylinder_adv_front.png}
        \caption{\en cylinder, Advancing front}
        \label{fig:cylinder_adv_front}
    \end{subfigure}
\end{figure}

Σε δεδομένα με μη σταθερή πυκνότητα (σχήμα \ref{fig:cylinder_adv_front})
το αποτέλεσμα έχει γενικά κακή τοπολογία, υπάρχουν πολύ μικρά και πολύ μεγάλα τρίγωνα,
υπάρχουν μικρές γωνίες, δεν υπάρχει ομαλότητα και ροή.

\begin{figure}
    \begin{subfigure}{0.3\textwidth}
        \centering
        \includegraphics[width=0.9\linewidth]{building_simple_points.png}
        \caption{\en building point set}
        \label{}
    \end{subfigure}
    \begin{subfigure}{0.3\textwidth}
        \centering
        \includegraphics[width=0.5\linewidth]{building_adv_front_jagged.png}
        \caption{Η γωνία του \en reconstruction \gr από το \en point set. \gr
                 }
        \label{fig:adv_front_jagged}
    \end{subfigure}
    \begin{subfigure}{0.3\textwidth}
        \centering
        \includegraphics[width=\linewidth]{jagged_correction.png}
        \caption{Διόρθωση}
        \label{fig:correction}
    \end{subfigure}
    \caption{}
    \label{fig:adv_problem}
\end{figure}


Ένα άλλο πρόβλημα αυτής της μεθόδου φαίνεται στο σχήμα \ref{fig:adv_problem}.
Στα αριστερά φαίνεται το \en point cloud \gr το οποίο δεν είχε θόρυβο,
στο κέντρο το \en mesh \gr που δημιουργήθηκε, σε μια από τις γωνίες.

Παρότι δεν υπήρχε θόρυβος στα δεδομένα, η γωνία εμφανίζεται \say{κομμένη}.
Για να διατηρηθεί σωστά η γεωμετρία θα έπρεπε για γίνει π.χ. η σύνδεση που φαίνεται 
με κόκκινο στο σχήμα \ref{fig:correction} και οι υπόλοιπες απαραίτητες ακμές για 
να την στηρίξουν. 

Για μεγαλύτερη ακρίβεια, θα έπρεπε οι κόμβοι που μοντελοποιούν τη γωνία να βρίσκονται 
\emph{ακριβώς} πάνω στην θεωρητική γωνία. Έτσι θα έπρεπε είτε να ανιχνευτούν οι κόμβοι 
που έχουν τις πιο εξωτερικές συντεταγμένες και να γίνουν οι ενώσεις (όπως η κόκκινη 
γραμμή στο σχήμα \ref{fig:correction}) μόνο μεταξύ τους, οπότε θα υπήρχαν δυσανάλογα
μεγάλα τρίγωνα στις γωνίες, ή να δημιουργηθούν καινούργιοι κόμβοι πάνω στην γωνία.

Παρόμοιο πρόβλημα βλέπουμε στα αυτιά της γάτας (σχήμα \ref{fig:kitten_adv_front}).

\begin{itemize}
    \item Αυτό παρουσιάζει ενδιαφέρον για περαιτέρω μελέτη.
\end{itemize}


Στο σχήμα \ref{fig:adv_front_noise} είναι το αποτέλεσμα σε δεδομένα με θόρυβο.
Όπως φαίνεται, δεν καταπολεμείται χωρίς προεπεξεργασία.

\begin{figure}
    \centering
    \includegraphics[width=0.5\linewidth]{adv_front_noise.png}
    \caption{}
    \label{fig:adv_front_noise}
\end{figure}




\section{\en Poisson}


Η μέθοδος \en Poisson \gr προσπαθεί να λύσει το πρόβλημα παρουσία θορύβου, 
επομένως είναι απαραίτητο να δημιουργηθούν νέοι κόμβοι για το \en mesh, \gr 
έναντι των σημείων του \en point cloud. \gr 
Λειτουργεί για \en watertight \gr κλειστά σχήματα.

Παίρνει το όνομά της διότι προσεγγίζει τη λύση μιας εξίσωσης \en Poisson \gr 
ως εξής:

Έστω ένα αντικείμενο που χωρίζει τον χώρο σε εσωτερικό και εξωτερικό  και έστω ότι γνωρίζουμε την κλίση του σε ορισμένα σημεία της επιφάνειας
(έχουμε \en normal vectors). \gr
Ορίζουμε την συνάρτηση $x$ ώς $\frac{1}{2}$ στο εσωτερικό και $-\frac{1}{2}$
στο εξωτερικό. Τότε η κλίση της συνάρτησης είναι 0 παντού εκτός από την επιφάνεια 
που θέλουμε να βρούμε, όπου είναι άπειρη και δείχνει προς το εσωτερικό. 
Για να αποκτήσει πραγματική τιμή (ίση με την τιμή των \en normals) \gr εφαρμόζουμε 
συνέλιξη με κάποια \en smoothness \gr συνάρτηση. Θέλουμε να εκτιμήσουμε την 
συνάρτηση $x$ από τα δεδομένα.

Αν $\mathbf{V}$ είναι η πραγματική κλίση, τότε η συναρτησιακή:
\[E(x) = \int || \nabla x(p) - \mathbf{V}(p) ||^2dp\]
δηλώνει την απόσταση της εκτίμησης $x$ από την επιθυμητή συνάρτηση.

Προσεγγίζουμε την συνάρτηση $x$ ελαχιστοποιώντας τη συναρτησιακή, ή ισοδύναμα \en (Euler-Lagrange) \gr λύνοντας
την εξίσωση \en Poisson: \gr 
\[\Delta x = \nabla \cdot \mathbf{V}\]


Γίνεται διακριτοποίηση του χώρου και λύνεται το σύστημα που προκύπτει από τη 
διατύπωση \en Galerkin \gr της παραπάνω διαφορικής για να εκτιμηθεί η $x$.
Έπειτα χρησιμοποιείται η $x$ για τη δημιουργία του \en mesh. \gr

Περισσότερες πληροφορίες στο φάκελο με τα \en papers (Screened Poisson). \gr 

\subsection{Κάποια αποτελέσματα}

Εδώ χρησιμοποιώ το \en MeshLab \gr για να παράγω αποτελέσματα.
Τα \en normals \gr δημιουργούνται από το \en point set \gr με τη μέθοδο 
του \en MeshLab. \gr 

\begin{figure}
    \begin{subfigure}{0.5\textwidth}
        \centering
        \includegraphics[width=0.5\linewidth]{kitten_Poisson.png}
        \caption{γάτα, \en Poisson}
        \label{}
    \end{subfigure}
    \begin{subfigure}{0.5\textwidth}
        \centering
        \includegraphics[width=0.5\linewidth]{kitten_Poisson_wireframe.png}
        \caption{γάτα \en Poisson, edges}
        \label{fig:kitten_Poisson}
    \end{subfigure}
\end{figure}


Για \en smooth \gr επιφάνειες, όπως η γάτα στο σχήμα \ref{fig:kitten_Poisson}
τα αποτελέσματα είναι πολύ καλά.
Εδώ δεν υπάρχει το πρόβλημα που είχε η μέθοδος \en advancing front \gr 
με τα αυτιά της γάτας.
Η τοπολογία όμως του παραγόμενου \en mesh \gr δεν είναι καλή (τρίγωνα με 
πολύ μικρές γωνίες)

\begin{figure}
    \centering
    \includegraphics[width=0.6\linewidth]{building_large_Poisson_default.png}
    \caption{}
    \label{fig:building_Poisson}
\end{figure}


Σε επιφάνειες με επίπεδα και γωνίες, όπως κτήρια (σχήμα \ref{fig:building_Poisson})
τα αποτελέσματα δεν είναι τόσο καλά. Στο κτήριο του σχήματος υπάρχουν προβλήματα
λόγω \say{κακών} δεδομένων (κενά, έντονος θόρυβος).


\begin{figure}
    \begin{subfigure}{0.5\textwidth}
        \centering
        \includegraphics[width=0.6\linewidth]{ball_Poisson.png}
        \caption{\en reconstructed mesh}
        \label{}
    \end{subfigure}
    \begin{subfigure}{0.5\textwidth}
        \centering
        \includegraphics[width=0.6\linewidth]{ball_Poisson_topology.png}
        \caption{Τοπολογία}
        \label{}
    \end{subfigure}
    \caption{}
    \label{fig:ball_Poisson}
\end{figure}

Σε δεδομένα με λίγο θόρυβο (σχήμα \ref{fig:ball_Poisson}) η μέθοδος \en Poisson \gr 
εξομαλύνει τον θόρυβο, αλλά κάνει επίσης \en smooth \gr στις 
γωνίες όπου χάνεται η ακμή. 

Ένα πρόβλημα, επομένως, της μεθόδου είναι το πρόβλημα της διατήρησης της ακμής 
η οποία χάνεται λόγω της συνέλιξης που προαναφέρθηκε (περιγράφεται στο \en paper \gr
.. ).


\begin{figure}
    \begin{subfigure}{0.5\textwidth}
        \centering
        \includegraphics[width=0.6\linewidth]{torus_500_points.png}
        \caption{\en points}
        \label{}
    \end{subfigure}
    \begin{subfigure}{0.5\textwidth}
        \centering
        \includegraphics[width=0.6\linewidth]{torus_500_Poisson.png}
        \caption{\en reconstructed mesh}
        \label{}
    \end{subfigure}
    \vfill
    \begin{subfigure}{0.5\textwidth}
        \centering
        \includegraphics[width=0.6\linewidth]{torus_40_points.png}
        \caption{\en points}
        \label{}
    \end{subfigure}
    \begin{subfigure}{0.5\textwidth}
        \centering
        \includegraphics[width=0.6\linewidth]{torus_40_Poisson.png}
        \caption{\en reconstructed mesh}
        \label{}
    \end{subfigure}
    \vfill
    \begin{subfigure}{0.5\textwidth}
        \centering
        \includegraphics[width=0.6\linewidth]{torus_10_points.png}
        \caption{\en points}
        \label{}
    \end{subfigure}
    \begin{subfigure}{0.5\textwidth}
        \centering
        \includegraphics[width=0.6\linewidth]{torus_10_Poisson.png}
        \caption{\en reconstructed mesh}
        \label{}
    \end{subfigure}
    \caption{Μέθοδος \en Poisson \gr σε αντικείμενο με τρύπα \en (torus) \gr για διαφορετικές 
    πυκνότητες σημείων στα δεδομένα}
    \label{}
\end{figure}

Στο σχήμα \ref{fig:dragon_poisson} βλέπουμε το \en reconstruction \gr για το \en point cloud \gr ενός δράκου. 
Στα \en scales \gr του δράκου έχει χαθεί η ακμή που τα διαχωρίζει μεταξύ τους. 


\begin{figure}
    \begin{subfigure}{0.5\textwidth}
        \centering
        \includegraphics[width=0.8\linewidth]{dragon_scales_points.png}
        \caption{Δεδομένα}
        \label{}
    \end{subfigure}
    \begin{subfigure}{0.5\textwidth}
        \centering
        \includegraphics[width=\linewidth]{dragon_scales_Poisson.png}
        \caption{Αποτέλεσμα}
        \label{}
    \end{subfigure}
    \caption{\en Reconstruction \gr δράκου.}
    \label{fig:dragon_poisson}
\end{figure}





\section{\en Kinetic}



Η μέθοδος αυτή δέχεται ως είσοδο \en point sets, \gr με \en normals \gr που δείχνουν προς τα έξω
και ένα σύνολο επιπέδων τα οποία έχουν ανιχνευθεί από τα σημεία.
Ως μηδενικό βήμα απλοποιεί κάθε ένα από τα επίπεδα ως το κυρτό περίβλημα των σημείων που βρίσκονται εντός του.
Έπειτα κάνει τις απαραίτητες τομές ώστε τα επίπεδα να μην τέμνονται μεταξύ τους.

Το κύριο βήμα του αλγορίθμου \en (kinetic partitioning) \gr διαμερίζει τον υπολογιστικό χώρο \en (bounding box \gr των δεδομένων) 
σε (κυρτά) πολύεδρα μεγαλώνοντας σε κάθε βήμα κάθε (κυρτό πλέον) επίπεδο μέχρι να 
ανιχνεύσει σύγκρουση, όπου μπορεί είτε να σταθεροποιήσει τους κόμβους στο σημείο 
της σύγκρουσης, είτε να αλλάξει την κατεύθυνση της κίνησής τους, είτε ναι διαμερίσει 
τα επίπεδα. Όταν έχουν σταθεροποιηθεί πλέον όλοι οι κόμβοι, έχουμε το διαμερισμό 
του χώρου σε πολύεδρα. 

Έπειτα ο αλγόριθμος αποφασίζει ποια πολύεδρα ανήκουν στο εσωτερικό και ποια στο 
εξωτερικό του αντικειμένου:
Θέτει για κάθε πολύεδρο μία τιμή $x_i = \{ in, out \}$
ελαχιστοποιώντας την:

\[U(\mathbf{x}) = D(\mathbf{x}) + \lambda V(\mathbf{x})\]

με 
\[D(\mathbf{x}) = \frac{1}{|I|} \sum_{i \in C} \sum_{p \in I_i} d_i(p,x_i)\]


όπου $C$ το σύνολο των πολυέδρων, $I_i$ τα σημεία που βρίσκονται 
\en(inlier) \gr στο πολύεδρο $i$ και η $d(p,x_i)$ είναι 1 όταν το \en normal vector 
\gr του σημείου $p$ δείχνει προς το εσωτερικό σύμφωνα με το $x_i$, και

\[V(\mathbf{x}) = \frac{1}{A} \sum_{i \sim j} a_{ij} \cdot 1_{\{x_i \neq x_j\}}\]


όπου με $\sim$ συμβολίζουμε γειτονικά πολύεδρα, $a_{ij}$ το εμβαδόν της κοινής 
πλευράς και $A$ ένα \en normalization factor \gr ίσο με το άθροισμα όλων των εμβαδών 
της διαμέρισης.

Ο όρος $U(\mathbf{x})$ είναι ελάχιστος για την καλύτερη προσέγγιση εσωτερικού και  
εξωτερικού, και ο όρος $V(\mathbf{x})$ υπάρχει για να απλοποιεί το σχήμα και να 
αποφευχθούν τα \en zig-zag. \gr 

Στην περίπτωση του \en CGAL \gr είναι ορισμένα ως

\[U(\mathbf{x}) = (1 - \lambda) D(\mathbf{x}) + \lambda V(\mathbf{x})\]

με 
\[D(\mathbf{x}) = \sum_{i \in C} \sum_{p \in I_i} d_i(p,x_i)\]
και 
\[V(\mathbf{x}) = \frac{1}{A} \sum_{i \sim j} a_{ij} \cdot 1_{\{x_i \neq x_j\}}\]

Η παραπάνω ελαχιστοποιείται με \en min-cut. \gr 

Αφού γίνει αυτό, επιστρέφεται το σύνολο των επιπέδων που έχουν μία πλευρά στο 
εσωτερικό και μια στο εξωτερικό του αντικειμένου ως το \en reconstructed mesh. \gr 

Η ανίχνευση των επιπέδων γίνεται με \en region growing, \gr αλλά μπορεί να γίνει 
και με \en RANSAC. \gr 

Περισσότερες λεπτομέρειες μπορείτε να βρείτε στο \en paper \gr (το έχω συμπεριλάβει στα αρχεία).


\subsection{Παράμετροι}

\en \textbf{Shape Detection}

\begin{itemize}
    \item maximum\_distance : The maximum distance of a point to a plane.
    \item maximum\_angle : The maximum angle in degrees between the normal associated with a point and the normal of a plane.
    \item minimum\_region\_size : The minimum number of points a region must have.
    \item k\_neighbors : defines the number K of nearest neighbors of the query point.
\end{itemize}

\textbf{Shape Regularization}

\begin{itemize}
    \item regularize\_parallelism : indicates whether parallelism should be regularized or not
    \item regularize\_coplanarity : indicates whether coplanarity should be regularized or not 
    \item regularize\_orthogonality : indicates whether orthogonality should be regularized or not 
    \item regularize\_axis\_symmetry : indicates whether axis symmetry should be regularized or not 
    \item maximum\_offset : maximum allowed orthogonal distance between two parallel planes such that they are considered to be coplanar
    \item angle\_tolerance : maximum allowed angle in degrees between plane normals used for parallelism, orthogonality, and axis symmetry 
    \item symmetry\_direction : an axis for symmetry regularization 
\end{itemize}

\textbf{Kinetic Space Partition}

\begin{itemize}
    \item k : the maximum number of intersections that can occur for a polygon before its expansion stops
    \item reorient\_bbox : Setting reorient\_bbox to true aligns the x-axis of the bounding box with the direction of the largest variation in horizontal direction of the input data while maintaining the z-axis
    \item bbox\_dilation\_ratio :  By default the size bounding box of the input data is increased by 10\% to avoid that input polygons are coplanar with the sides of the bounding box.
    \item max\_octree\_node\_size : A kinetic partition is split into 8 subpartitions using an octree if the number of intersecting polygons is larger than specified. The default value is 40 polygons.
    \item max\_octree\_depth : Limits the maximum depth of the octree decomposition. A limitation is necessary as arbitrary dense polygon configurations exist, e.g., a star. The default value is set to 3.
\end{itemize}

\gr

\subsection{Αποτελέσματα}

\begin{figure}
    \begin{subfigure}{0.5\textwidth}
        \centering
        \includegraphics[width=\linewidth]{cube_extr_kinetic.png}
        \caption{Άρτιο αποτέλεσμα}
        \label{}
    \end{subfigure}
    \begin{subfigure}{0.5\textwidth}
        \centering
        \includegraphics[width=\linewidth]{icosahedron_Kinetic.png}
        \caption{Εικοσάεδρο. Καλό αποτέλεσμα εκτός από τις γωνίες.}
        \label{}
    \end{subfigure}
    \caption{}
    \label{}
\end{figure}


Η μέθοδος είναι ευαίσθητη στις παραμέτρους και χρειάζεται αρκετές δοκιμές για να παρθεί καλό αποτέλεσμα.
Για μεγάλα \en datasets \gr αυτό αποτελεί πρόβλημα καθώς μπορεί κάθε δοκιμή να τρέχει για ώρες.

Δοκίμασα κάποια 
δεδομένα που δόθηκαν από τον συγγραφέα που πρότεινε τη μέθοδο με τις παραμέτρους 
που προτείνουν στη \en CGAL: \gr 

\begin{figure}
    \begin{subfigure}{0.5\textwidth}
        \centering
        \includegraphics[width=0.8\linewidth]{Hilbert_cube_points.png}
        \caption{Δεδομένα}
        \label{}
    \end{subfigure}
    \begin{subfigure}{0.5\textwidth}
        \centering
        \includegraphics[width=\linewidth]{Hilbert_cube_Kinetic.png}
        \caption{Αποτέλεσμα}
        \label{}
    \end{subfigure}
    \caption{\en Hilbert cube}
    \label{fig:Hilbert_cube}
\end{figure}

Για το παράδειγμα \en Hilbert\_Cube \gr (σχήμα \ref{fig:Hilbert_cube})
πήρα τέλεια αποτελέσματα (με μόνη εξαίρεση κάποιους αχρείαστους κόμβους σε σενευθειακά τμήματα). 

Με δικά μου συνθετικά δεδομένα που παρήγαγα από τη σφαίρα του σχήματος \ref{fig:dende_sphere}
(αριστερά) παρήγαγα (στο κέντρο) με \en Poisson \gr και (δεξιά) με \en Kinetic. \gr
Η \en Poisson \gr παράγει πυκνή γεωμετρία με \en smoothed \gr ακμές, ενώ 
η \en Kinetic \gr καταφέρνει να \say{πιάσει} κάθε πλευρά σωστά, με εξαίρεση 
τις πολύ μικρές που βρίσκονται στους πόλους. Παρουσιάζει πολύ καλό αποτέλεσμα,
αν και είναι σημαντικά πιο αργή από την \en Poisson. \gr 



\begin{figure}
    \begin{subfigure}{0.3\textwidth}
        \centering
        \includegraphics[width=0.8\linewidth]{sphere.png}
        \caption{Αρχική}
        \label{}
    \end{subfigure}
    \begin{subfigure}{0.3\textwidth}
        \centering
        \includegraphics[width=\linewidth]{dense_sphere_Poisson.png}
        \caption{\en Poisson}
        \label{}
    \end{subfigure}
    \begin{subfigure}{0.3\textwidth}
        \centering
        \includegraphics[width=\linewidth]{dense_sphere_Kinetic.png}
        \caption{\en Kinetic}
        \label{}
    \end{subfigure}
    \caption{\en Sphere consisting of planar shapes}
    \label{fig:dende_sphere}
\end{figure}

Στο σχήμα \ref{fig:foam_box} φαίνεται ένα παράδειγμα με δεδομένα που περιέχουν \en outliers \gr και 
κενά. Η μέθοδος κατάφερε να ανιχνεύσει τη γεωμετρία, αλλά με αρκετά προβλήματα. Δύο πλευρές είναι 
\en degenerate \gr (είναι το \en warning \gr που μου δίνει το \en meshLab, \gr μπορεί να σημαίνει πολλά, γενικά 
είναι ότι δεν είναι καλώς ορισμένη), είναι οι δύο πλευρές που δεν εμφανίζονται στην αναπαράσταση (στο κέντρο). 

Στο \en manual \gr της \en cgal \gr παίρνουν καλύτερα αποτελέσματα 
αλλα δεν καταφέρνω να τα αναπαράγω.


\begin{figure}
    \begin{subfigure}{0.3\textwidth}
        \centering
        \includegraphics[width=0.8\linewidth]{foam_box_points.png}
        \caption{\en point set}
        \label{}
    \end{subfigure}
    \begin{subfigure}{0.3\textwidth}
        \centering
        \includegraphics[width=\linewidth]{foam_box_mesh.png}
        \caption{\en reconstructed mesh (Kinetic)}
        \label{}
    \end{subfigure}
    \begin{subfigure}{0.3\textwidth}
        \centering
        \includegraphics[width=\linewidth]{foam_box_together.png}
        \caption{\en both}
        \label{}
    \end{subfigure}
    \caption{\en Foam Box}
    \label{fig:foam_box}
\end{figure}



Η μέθοδος φαίνεται αδύνατη να ανακατασκευάσει έναν απλό κύβο!




\subsection{Τεχνικά Προβλήματα}

Στα περισσότερα αποτελέσματα της μεθόδους αυτής υπάρχουν \en degeneracies, \gr π.χ. \en duplicate vertices, \gr δηλαδή δύο κόμβοι που βρίσκονται στο ίδιο ακριβώς σημείο, κ.α.

Κάποια από αυτά λύνονται εύκολα με \en post processing, \gr (π.χ. τα \en duplicates, \gr ή οι αχρείαστοι κόμβοι). Άλλα δεν γνωρίζω
(όπως τα \en degenerate faces \gr που δε ξέρω ακριβώς γιατί προκύπτουν).

Τέλος, σε κάποια \en datasets \gr ο αλγόριθμος κρασάρει και δεν παράγει αποτέλεσμα. Στο \en manual \gr της \en CGAL \gr για τα ίδια 
\en datasets \gr παράγουν αποτελέσματα, όποτε μάλλον κάτι γίνεται λάθος από μεριάς μου.

\begin{figure}
    \begin{subfigure}{0.3\textwidth}
        \centering
        \includegraphics[width=0.8\linewidth]{Kin_degenerecies_ds1.png}
        \caption{Άσκοποι κόμβοι σε συνευθειακά τμήματα}
        \label{}
    \end{subfigure}
    \begin{subfigure}{0.3\textwidth}
        \centering
        \includegraphics[width=\linewidth]{Kin_degenerecies_ds2.png}
        \caption{Περίεργες συνδέσεις σε ορθές γωνίες}
        \label{}
    \end{subfigure}
    \begin{subfigure}{0.3\textwidth}
        \centering
        \includegraphics[width=\linewidth]{foam_box_degeneracies.png}
        \caption{Πλευρές που δεν οπτικοποιήθηκαν }
        \label{}
    \end{subfigure}
    \caption{Προβλήματα}
    \label{fig:problems}
\end{figure}

\begin{figure}
    \begin{subfigure}{0.5\textwidth}
        \centering
        \includegraphics[width=\linewidth]{lans_Kinetic.png}
        \caption{Σε ένα κατά τα άλλα καλό αποτέλεσμα, υπάρχουν επικαλυπτόμενες πλευρές (σκιασμένες με γκρι)}
        \label{}
    \end{subfigure}
    \caption{\en Degeneracies}
    \label{fig:degeneracies}
\end{figure}






\section{\en Pre-processing}

Δεν χρησιμοποίησα σε καμία από τις παραπάνω μεθόδους \en pre-processing \gr 
των δεδομένων, το οποίο είναι αρκετά σημαντικό για τις περισσότερες. Τα αποτελέσματα
επομένως δεν είναι πλήρη.





\end{document}

